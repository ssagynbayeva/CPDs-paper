% Define document class
\documentclass[twocolumn]{aastex631}
\usepackage{showyourwork}
% \newcommand{\cntext}[1]{\begin{CJK}{UTF8}{bkai}#1\ignorespacesafterend\end{CJK}} % Typeset Chinese characters
% \usepackage{CJK}
\usepackage{xeCJK}

% Begin!
\begin{document}
% \begin{CJK*}{UTF8}{gbsn}
% Title
\title{Global simulation of thermal torques}

% Author list
\author[0000-0002-6650-3829]{Sabina Sagynbayeva}
\email{sabina.sagynbayeva@stonybrook.com}
\affiliation{Department of Physics and Astronomy, Stony Brook University, Stony Brook, NY 11794, USA}
\affiliation{Center for Computational Astrophysics, Flatiron Institute, 162 Fifth Avenue, New York, NY 10010, USA}

% \author[0000-0002-2624-3399]{Yan-Fei Jiang(姜燕飞)}
% \affiliation{Center for Computational Astrophysics, Flatiron Institute, 162 Fifth Avenue, New York, NY 10010, USA}

\author[0000-0001-9222-4367]{Rixin Li (李日新)}
\affiliation{Center for Astrophysics and Planetary Science, Department of Astronomy, Cornell University, Ithaca, NY 14853, USA}

\author[0000-0001-5032-1396]{Philip J. Armitage}
\affiliation{Department of Physics and Astronomy, Stony Brook University, Stony Brook, NY 11794, USA}
\affiliation{Center for Computational Astrophysics, Flatiron Institute, 162 Fifth Avenue, New York, NY 10010, USA}


% Abstract with filler text
\begin{abstract}
   
\end{abstract}

% Main body with filler text
\section{Introduction} 
\label{sec:intro}
Circumplanetary disks (CPDs) are believed to be a natural byproduct of the planet formation process, arising from the vertical infall of gas from the 
protoplanetary disk onto the growing protoplanet \citep{Ward2010}. Simulations have shown that CPDs exhibit complex dynamics affected by viscosity, 
thermal structure, magnetic fields, and turbulence \citep{Fujii2011,Gressel2013,Szul2014}. Early numerical studies utilized 
two-dimensional hydrodynamic models to demonstrate disk formation and characterize basic properties like size and density distribution \citep{Dangelo2003}. 
Later work incorporated three-dimensional simulations to better capture non-axisymmetric flows and vertical disk structure \citep{Machida2008}. 

A major limitation of early models was the use of simplified thermodynamic treatments like locally isothermal equations of state. Recent research has 
focused on more realistic energy equations to determine the effects of radiation transport, viscous heating, and radiative cooling on CPD structure and 
evolution \citep{Szul2016,Zhu2016}. Thermal physics can alter turbulence and angular momentum transport in the disk, which may affect accretion onto the planet. 
Other areas of active research include studying magnetic field amplification through MRI-driven turbulence \citep{Fujii2014}, exploring disk chemistry 
and composition \citep{Fujii2017}, and analyzing impacts of planet mass and orbital properties \citep{Szul2017a}.

CPDs have also long been hypothesized to be the birthplaces of regular moons around giant planets \citep{Lunine1982,Canup2006}. 
However, the efficiency of satellite formation within CPDs depends on several factors that remain poorly constrained. Early analytic work suggested disks 
may be too compact and short-lived to readily form large moons \citep{Goodman2001}. But recent hydrodynamic simulations indicate CPDs can exhibit 
extended disks over 100s of planetary radii and lifetimes over 1000 orbits \citep{Szul2016,Fujii2017}.

Higher resolution 3D MHD models reveal turbulence and instabilities that could concentrate solids and aid satellite growth. 
Analysis of pebble accretion shows Galilean moon formation may be possible in CPDs around Jupiter-like planets \citep{Shibaike2017,Ronnet2018}. 


\section{Methods}
\label{sec:methods}

\subsection{Disk model}
Isentropic disk model from \citet{Fung_2017}.
The simulations are performed in spherical coordinates, where $r$, $\phi$, and $\theta$ denote the usual radial, azimuthal, and polar coordinates, respectively. 

\begin{equation}\label{eq:potential}
\begin{aligned}
     \Phi = & -\frac{G(M_*+M_p)}{1+q} \bigg[\frac{1}{r} \\
     & + \frac{q}{\sqrt{r^2+R_p^2-2rR_p\cos{\phi'}+\epsilon^2}} - \frac{qR\cos{\phi'}}{R_p^2}\bigg] \\
\end{aligned}
\end{equation}

In eq. \ref{eq:potential}, $q = M_p/M_*$ is the mass ratio of a protoplanet to a star, $\epsilon$ is the smoothing length of the planet's potential, and $\phi'= \phi-\phi_p$ denotes the azimuthal separation from the planet. $GM=1$, $R_p=1$.

Keplerian velocity $v_k=\sqrt{\frac{GM}{r}}$ and keplerian frequency $\Omega_k=\sqrt{\frac{GM}{r^3}}$ equal to 1 at the planet's orbit.

$v_p$ - planet's orbital speed. 
$q = 1.5\times 10^{-5} \approx 5 M_{\oplus}$

The planet is introduced to the disk gradually, where its mass increases to the desired value over the first orbit.

The simulation domain spans $0.65R_p$ to $1.35R_p$ in the radial, and the full $2\pi$ in azimuth.

The Hill radius of a planet is:
\begin{equation}\label{eq:Hill}
    r_H=R_p\left(\frac{q}{3}\right)^{\frac{1}{3}}
\end{equation}

$r_s$ is a small fraction of $r_H$, between 3\% to 10\% of $r_H$.

The isentropic equation of state:
\begin{equation}\label{eq:pressure}
    p = \frac{c_0^2\rho_0}{\gamma}\left(\frac{\rho}{\rho_0}\right)^\gamma
\end{equation}

In Eq. \ref{eq:pressure}, $c_0$ is the isothermal sound speed when $\rho=\rho_0=1$. 
The disk's scale height $H = 0.05$ scales the sound speed as $c_0 = H\Omega=0.05$.

\subsubsection{Initial conditions}
For hydrostatic equilibrium:
\begin{equation}\label{eq:density}
\begin{aligned}
     \rho = & \rho_0 \bigg[\left(\frac{R}{R_p}\right)^{(-\beta+\frac{3}{2})\frac{2(\gamma-1)}{\gamma+1}} \\ 
    & - \frac{GM(\gamma-1)}{c_0^2}\left(\frac{1}{R}-\frac{1}{r}\right)\bigg]^{\frac{1}{\gamma-1}} \\
\end{aligned}
\end{equation}
$\beta=\frac{3}{2}$ defines the surface density profile $\Sigma\propto R^{-\beta}$.

The orbital frequency of the disk:
\begin{equation}\label{eq:omegadisk}
    \Omega = \sqrt{\Omega_k^2+\frac{1}{r\rho}\frac{\partial P}{\partial r}}
\end{equation}

$v_r=v_\theta=0$

\subsubsection{Grid setup}
We use spherical-polar grid with a nested static mesh refinement (SMR) around the location of the planet. The intervals that span in radial, polar, and azimuthal
directions are $[0.4, 2.5]$, $[9\pi/20, 11\pi/20]$, and $[0, 2\pi]$, respectively. Aiming for $50$ cells around the Hill's sphere, we came up with the 
effective resolution at the \emph{root} level: $168$ cells across the $r$-direction, $576$ cells across the $\phi$-direction, 
and $72$ cells across the $\theta$-direction.

Nested SMR that we implemented helped us to lower the resolution at the root level keeping high enough resolution around the Hill's 
sphere to resolve the CPD properly. The detailed description of the finest of the refined regions is shown on Table \ref{tab:smr}.
%
\begin{table}[]
    \label{tab:smr}
    \begin{tabular}{lll}
                   &                                 &          \\ \hline \hline
    Refinement \#1 & $9\pi/20 \rightarrow 11\pi/20$  & $\theta$ \\
                   & $0.8 \rightarrow 1.2$           & $r$      \\
                   & $\pm 0.2$                       & $\phi$   \\ \hline
    Refinement \#2 & $19\pi/40 \rightarrow 21\pi/40$ & $\theta$ \\
                   & $0.9 \rightarrow 1.1$           & $r$      \\
                   & $\pm 0.1$                       & $\phi$   \\ \hline
    Refinement \#3 & $39\pi/80 \rightarrow 41\pi/80$ & $\theta$ \\
                   & $0.95 \rightarrow 1.05$         & $r$      \\ 
                   & $\pm 0.05$                      & $\phi$  \\ \hline
    \end{tabular}
\end{table}
%
The definitions of the normalization units and the corresponding values are listed in Table \ref{tab:units}.
%
\begin{table}[]
    \label{tab:units}
    \caption{Normalization Units and Corresponding Values}
    \begin{tabular}{lll}
    \hline \hline
    Quantity            & Unit                  \\ \hline
    Length              & $r_p$                  \\
    Velocity            & $v_{K0}$                 \\
    Time                & $r_p/v_{K0}$             \\
    Density             & $\rho_0$                 \\
    Pressure            & $\rho_0 v_{K0}^2$             \\
    Mass accretion rate & $\rho_0 r_p^2 v_{K0}$               \\ \hline
    \end{tabular}
\end{table}
%
\subsubsection{Resolution}

\subsection{Athena++ simulations}

\section{Results}
\label{sec:results}

\section{Discussion} 
\label{sec:conclusion}

\bibliography{bib}
% \end{CJK*}
\end{document}
